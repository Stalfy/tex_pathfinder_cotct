\documentclass[main.tex]{subfiles}

\begin{document}
    Olrynth est maintenant à la recherche d'indices à propos de Gaedrenn Lamm depuis quelques semaines.
    Minutieux, il a pris l'habitude d'investiguer toutes les pistes qui s'offrent à lui sans égard pour leur crédibilité.
    Cependant, les informations qu'Olrynth a obtenu jusqu'à présent sont souvent imprécises ou incomplètes, ce qui l'empêchait de progresser.
    Malgré ses recherches infructueuses, il restait fidèle à ses habitudes de combattant.
    Tous les matins, il se lèvait aux premières lueurs du soleil pour s'entraîner et peaufiner ses techniques de combat.
    Pour ne pas endommager inutilement son arme de prédilection, ses poings, Olrynth a pris l'habitude de les bander avant de s'entraîner.
    Un matin, alors qu'il défaisait ses bandages après son entraînement pour les nettoyer, il remarqua un relief inhabituel dans le bandage de sa main droite.
    Il trouva dans un des tours de son bandage ce qui semblait être une carte de tarot.\\
    \\
    Il était possible de déterminer en un seul coup d'oeil que cette carte provenait d'un jeu de tarot de très grande qualité.
    La teinte de brun présente sur le dos de la carte montrait que le jeu avait été fait à partir de bois de frêne.
    Les biens produits à partir de ce bois inexistant sur le continent de Varisia étaient évalués à des centaines de pièces d'or.
    Chaque côté de cette face présentait un motif de vague de couleur cacao.
    Entre les sommêts des vagues, un petit point de couleur beige ambré avait été dessiné avec une précision hors pair.
    Le bas des vagues était relié par un rentangle ne présentant aucune imperfection dans son trait.
    Dans chaque coin de l'intérieur du rectangle, un croissant de lune pointant vers le centre de la carte était déssiné.
    Au centre de la carte se trouvait une pleine lune entourée par plusieurs autres croissants de lune, tous de la même couleur beige.
    Cette carte avait également été lustrée pour préserver l'éclat des couleurs du motif.
    Le dos de la carte ne comprenait qu'une seule inscription.\\
    \\
    Olrynth Bloodbrow.\\
    \\
    Son nom complet était inscrit au centre de la pleine lune sur le dos de la carte.
    Cette découverte le laissa perplexe.
    Depuis son arrivée à Korvosa, il n'avait jamais divulgué son nom complet pour éviter d'être retrouvé par son clan.
    Après une courte hésitation, il décida de retourner la carte.
    Sur l'autre de face de la carte, d'une couleur ivoire éclatante, se trouvait un bref message :\\
    \\
    \textit{
        Je sais ce que Gaedrenn vous à fait et où il se cache.
        Comme vous et comme plusieurs autres, je cherche à me venger de cet ignoble personnage.
        Il doit payer pour ses actions.
        Rejoignez-moi à cette addresse.
        Nous pourrons discuter plus amplement en personne.
    }\\
    \\
    Depuis le début de ses recherches, Olrynth n'avait jamais rencontré un informateur qui prétendait connaître la cachette de Gaedrenn Lamm.
    Après une courte réflexion, il décida d'aller vérifier cette nouvelle piste.
    Toute information le rapprochant de Gaedrenn Lamm était bonne à prendre.
    L'addresse indiquée sur la carte de tarot le mena vers un quartier plus défavorisé de Korvosa : Midland.
    Situé près de la rivière Jeggare, on retrouvait dans ce quartier quelques tavernes où les pêcheurs de la ville terminaient leurs soirées.
    Les bâtiments, majoritairement délabrés, étaient partagés par plusieurs familles qui n'arrivaient pas à payer la taxe municipale.
    Le long des quais, plusieurs bâtiments spécialisés n'étaient plus en fonction.
    Ce quartier faisait partie de ceux contrôlés par Gaedrenn, selon les informations qu'Olrynth avait récupérées.
    Après une courte recherche, Olrynth trouva l'addresse indiquée sur la carte.\\
    \\
    Contrairement aux autres bâtiments du secteur, celui-ci était en excellente condition.
    Les murs extérieurs, en pierre, ne présentaient aucun défauts.
    Le toit de tuiles semblait avoir était reconstruit récemment pour rajeunir l'allure du bâtiment.
    Aux fenêtres, des rideaux violets constellés d'étoiles dorées empêchaient de regarder à l'intérieur.
    La porte, remise à neuve, s'ouvrit d'elle-même lorsqu'il s'en approcha.
    Une fois à l'intérieur, la porte se referma derrière lui.
    Il se retrouva dans une pièce ayant pour seul mobilier une grande table circulaire et quelques chaises.
    Sur la table, une étoile à cinq pointes était dessinée à l'aide de peinture dorée.
    Les chaises étaient disposées de façon à ce que chaque chaise soit placée à une pointe de l'étoile.
    Aux murs, il vit des éléments typiques d'une maison de tarot, mais rien de notable.
    De l'autre côté de la pièce, il arriva à distinguer une porte cachée par une tapisserie donnant accès aux autres pièces du bâtiment.\\
    \\
    Quatre des cinq chaises étaient présentement occupées autour de la table.
    Immédiatement sur la gauche d'Olrynth se trouvait une chaise vide.
    Dans la deuxième chaise sur sa gauche était assise une humaine.
    Armée d'un sabre qu'elle portait à la taille, elle était protégée par une cuirasse de fer.
    Coupés à la hauteur des épaules, ses cheveux noirs se déposaient sur un foulard rouge protégeant son cou.
    Son expression faciale, qui montre une détermination à toute épreuve, trahissait son calme actuel.
    Sur sa droite, près d'Olrynth, se trouvait un halfelin et son loup.
    Plus petit qu'Olrynth, ce petit homme aux cheveux brun et aux oreilles pointues portait présentement une armure de cuir légère.
    Dans son dos était fixé une lance simple d'excellente qualité.
    La deuxième chaise sur sa droite était occupée par un gnome.
    Ce petit gnome, lui aussi affublé d'une armure de cuir, avait un carquois plein et un arc sur sa personne.
    Olrynth, quant à lui, était un nain à la barbe et aux cheveux roux.
    Il portait une petite tunique en cuir brun et aux manches coupées, révélant son impressionante musculature.
    Sous sa tunique, il portait un pantalon de coton dans lequel il avait rangé la carte.
    Finalement, droit devant Olrynth était assise une femme masquée.
    Il était impossible de distinguer ses traits, mais ses longs cheveux bruns dépaissaient du capuchon de sa robe longue et se déposaient doucement sur son torse.
    Devant elle, un jeu de tarot auquel appartenait la carte qu'Olrynth avait reçue était posé sur la table.\\
    \\
    \speech{Femme masquée}{
        Bonjour, Olrynth.
        Prenez place, nous vous attendions.
    }

    Olrynth s'approcha et pris place à table avec ces quatre autres personnages.
    Pendant qu'il prenait place, la femme masquée pris son jeu de tarot et le scinda en deux piles.
    En réponse à ce mouvement, la carte qu'Olrynth portait quitta la poche de son pantalon et se logea dans la pile de cartes que la femme tenait dans sa main droite.
    Trois autres cartes, détenues par les trois autres personnes à table, firent de même.
    Lorsque les cartes furent toutes dans sa main, la femme masquée referma le paquet et commença à le brasser distr  aitement alors qu'elle s'addressait au groupe.\\
    \\
    \speech{Femme masquée}{
        Merci d'être venus me rencontrer aujourd'hui.
        Je dois démeurer inconnue pour le moment sinon Gaedrenn me retrouvera et me tuera.
        Dans ce quartier, il fait respecter sa loi à l'aide de ses sbires, qu'il surnomme ses «little lambs».
        Ils sont ses bras, ses yeux et ses oreilles.
        Je vous ai réunis ici parce que nous avons tous une raison d'haïr Gaedrenn.
        Pour ma part, il a tué mon fils, Aeryn, lorsqu'il a tenté de récupérer ce jeu de tarot transmis de génératio  n en génération dans ma famille.
        Je me suis tournée vers la garde pour essayer de le faire arrêter, mais ils m'ont repoussé, comme tous les a  utres.
        Il a échappé à la loi durant des décennies et c'est maintenant que cette farce doit cesser.
        Le mot circule en ville qu'il doit être puni pour ses crimes.
    }
    \\
    Elle arrêta ensuite de brasser le jeu et tira des cartes, les groupant devant elle dans une position similaire a  ux chaises.
    Elle prit un moment pour observer les cartes, puis elle se tourna vers l'humaine sur sa droite.\\
    \\
    \speech{Femme masquée}{
        Elyn, je vois dans votre présent que la clarté de votre esprit vous permettra de prendre une bonne décision    dans un futur proche.
        Olrynth, vos cartes révèle que votre présent sera porteur d'une nouvelle camaraderie.
        Rothaul, votre passé montre que votre loyauté inflexible vous a déjà trahis.
        Cette trahison sera dans votre futur la clé d'une nouvelle destinée.
        Pojin, vos cartes disent que votre présence ici n'est pas un hasard.
        Vous saurez où trouvez Gaedrenn quand vos serez près de lui.
        Alors, qu'en dites-vous? Êtes-vous prêt à libérer Korvosa du joug de Gaedrenn Lamm?\\
    }
    \\
    \speech{Pojin}{Vous pouvez compter sur moi, gente damoiselle.}
    \\
    \speech{Rothaul}{Luc et moi ferons le nécessaire pour nous débarasser de ce vieux salopard.}
    \\
    \speech{Elyn}{Lamm ne mérite que la justice de mon sabre.}
    \\
    Olrynth, quant à lui, se contenta de hocher la tête.
    Malgré son exil, il décida de préserver son engagement ancestral, une marque de respect envers ses origines.\\
    \\
    \speech{Femme masquée}{
        Nous avons donc maintenant un ennemi commun.
        Vous pourrez le trouver dans la vieille poissonnerie du 17e quai.
        Il s'en sert présentement pour manufacturer du «fish slurry», une purée faite à partir de restes de poissons pourris.
        Les habitants du quartier n'ayant que peu de moyens, ils dépendent tous de lui pour se nourrir.
        C'est ignoble, mais lucratif.
        Rien ne l'empêchera d'allourdir un peu plus sa bourse.
        Je compte sur vous, braves guerriers.
    } 
    \\
    Après ce dernier encouragement, elle se leva, pris ses cartes de tarot et quitta par la porte cachée par une tapisserie derrière elle.
    Le groupe se consulta du regard.
    Ils avaient tous une raison d'haïr Gaedrenn Lamm, mais ils ne désiraient pas la communiquer avec des étranger.
    Olrynth, qui connaissait bien la ville, se leva et fit signe au groupe de le suivre.
    Il quitta ensuite le bâtiment et se dirigea vers le 17e quai.
    Les trois autres personnages se levèrent et lui emboîtèrent le pas.
    Ils étaient tous déterminés à en finir une fois pour toute.
\end{document}
