\documentclass[main.tex]{subfiles}

\begin{document}
    L'histoire des nains relate plusieurs conflits contre les clans de gobelins sur le continent de Varisia.
    Parmis ces conflits se trouve la célèbre « Croisade de Cuivre », une bataille menée par les nains pour contre les gobelins pour occuper Demdarahl, leur montagne sacrée.
    Durant cette guerre, le clan nain des haches d'acier se démarqua par son efficacité et son style révolutionnaire combinant prises et maniement de hache.
    Après la guerre, ce clan fut désigné pour protéger Demdarahl contre les invasions futures.
    À ce jour, la montagne a été attaquée dans un seul autre conflit majeur, les «Guerres de sang gobelin». 
    Dans ce conflit, ce style combinant une armure lourde et une hache afin de former une attaque et une défense redoutable, 
    en plus d'utiliser la hache pour déséquilibrer et faire perdre pied à son adversaire, permis aux nains de protéger à nouveau leur lieu sacré contre les envahisseurs.
    Après ce conflit, tous les moines nains du continent adoptaient ce style en y apportant très peu de modifications.\\
    \\
    Olrynth vécut une enfance paisible sous la protection des haches d'acier.
    Enfant unique d'un père bienveillant et d'une mère attentionnée, il a pu vivre une enfance paisible avec les autres nains du clan.
    À l'âge de 16 ans, comme le veulent les traditions du clan, Olrynth débuta son apprentissage du style ancestral de la hache d'acier.
    Passionné de nature, il devint rapidement l'un des meilleurs élèves de sa génération.
    À cause de sa maîtrise exceptionnelle des techniques de son style, il était vu de tous comme un génie.
    Il étudia ce style avec ferveur pendant les 40 prochaines années de sa vie.
    Il prit également l'engagement ancestral de ne parler qu'en cas d'extrême nécessité.
    Cependant, plus Olrynth étudiait ce style, plus il voyait des failles aux techniques ancestrales. 
    Il commença alors à développer secrètement un style faisant abstraction de l'armure lourde et de la hache. 
    Avec le style de la hache d'acier comme base, Olrynth développa plusieurs nouvelles techniques combinant esquives et contre-attaques.
    Un soir, alors qu'il raffinait ses techniques, il fut surpris par l'un de ses confrères qui s'empressa de rapporter ses actions au grand maître.\\
    \\
    La nouvelle circula rapidement à l'intérieur du clan. 
    La journée suivant cette dénonciation, Olrynth fut condamné à l'exil par son clan pour ses actions hérétiques et son manque de foi envers le style ancestral.
    Il quitta alors les montages Kodar où se trouvait Demdarahl et, ne sachant pas où aller, se dirigea vers le sud.
    Son voyage ne fut pas sans encombres. La nouvelle de son exil avait été transmise à tous les moines nains des régions avoisinantes.
    Aussitôt reconnu, Olrynth se faisait attaquer par des moines désireux de prouver que le style ancestral était meilleur que son nouveau style.
    À chaque bataille, les techniques révolutionnaires d'Olrynth lui octroyèrent la victoire sans recevoir une seule blessure.
    Il atteint éventuellement Urglin, mais ce village étant un village d'orcs, il n'y entra pas.
    Les nains varisiens sont réputés pour leur haine particulière des orcs et des gobelins, un sentiment réciproque.
    Il continua donc d'errer vers le sud jusqu'à ce que son nom ne soit plus connu.
    Après quelques semaines, il se trouvait devant les portes de Korvosa.\\
    \\
    Très peu de moines nains se trouvaient dans la cité de Korvosa. 
    Puisque son nom n'était pas connu dans cette ville, il décida de s'y établir.
    Sans le sous, il s'inscrit comme combattant dans une arène souterraine de Korvosa avec son nouveau style : le style du poing de granite.
    Il se fit rapidement une réputation et de plus en plus de combattants se présentaient à l'arène pour le défier.
    Ne refusant aucun combat, Olrynth demeurait invaincu dans l'arène.
    Malgré son contentement avec sa vie actuelle, Olrynth n'arrivait pas à guérir son mal du pays.
    C'est alors que quelqu'un lui offrit du «Frisson», une drogue qui lui permettrait de revivre en rêves toutes les années passées dans Demdarahl avec son clan.
    Au début, il n'en prenait que lorsqu'il n'avait pas de combats la journée suivante.
    Avec le temps, il commença à développer une dépendance à cette drogue et à en consommer de plus en plus fréquemment et en plus grande quantité.
    Un soir, son vendeur lui présenta un nouveau produit qui était prétendument plus puissant et qui permettait d'avoir des rêves plus lucides.
    Biaisé par son mal du pays et sa dépendance, Olrynth décida d'essayer ce nouveau produit.
    Ce soir là, il consomma la nouvelle substance et son corps réagit instantanément. 
    La dose était trop puissante pour son seuil de tolérance. 
    Il tomba dans un profond sommeil et son corps commença à frissonner et à trembler violemment.
    Il demeura inconscient dans cet état à l'intérieur de sa demeure pendant les 48 prochaines heures.\\
    \\
    Lorsqu'il se réveilla, il comprit qu'il avait miraculeusement survécu à cette surdose.
    Le soir venu, il se rendit à l'arène pour demander des explications à son vendeur.
    Lorsque ce dernier vit Olrynth, il prit peur et commença à fuir.
    Olrynth, en forme exceptionnelle à cause de ses combats réguliers, rattrapa le jeune sans peine.
    Après un bref «interrogatoire», Olrynth découvrit que le gestionnaire de l'arène, Gaedrenn Lamm, avait tenté de se débarrasser de lui puisque les combats menés par Olrynth ne lui rapportaient plus assez d'argent. 
    Il explosa immédiatement de rage.
    Pour un nain, une manoeuvre aussi sournoise pour l'éliminer était une offense à sa fierté. 
    Il avait maintenant un objectif, chose qu'il n'avait pas eu depuis des mois.
    Il commença par se tourner vers la garde, qui lui fit promptement comprendre qu'elle ne l'aiderait pas puisqu'elle avait de plus gros problèmes à régler.
    Olrynth décida donc d'arrêter les actions de Gaedrenn lui-même.
    Au fil de ses investigations, il remarqua également que le Frisson devenait de plus en plus utilisé par les jeunes combattants de l'arène, chose qu'il était incapable de tolérer.
    Olrynth est maintenant sur les traces de cet horrible criminel et il ne s'arrêtera que lorsque ce dernier ne pourra plus attaquer sa famille.
\end{document}
