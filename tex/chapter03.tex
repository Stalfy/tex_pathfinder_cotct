\documentclass[main.tex]{subfiles}

\begin{document}
    Après ce dernier encouragement, elle se leva, pris ses cartes de tarot et quitta par la porte cachée par une tapisserie derrière elle.
    Le groupe se consulta du regard.
    Ils avaient tous une raison d'haïr Gaedrenn Lamm, mais ils ne désiraient pas la communiquer avec des étranger.
    Olrynth, qui connaissait bien la ville, se leva et fit signe au groupe de le suivre.
    Il quitta ensuite le bâtiment et se dirigea vers le 17e quai.
    Les trois autres personnages se levèrent et lui emboîtèrent le pas.
    Ils étaient tous déterminés à en finir une fois pour toute.\\
    \\
    Lorsque le groupe arriva à la poissonnerie, ils décidèrent d'attendre le soir avant d'agir.
    Ils utilisèrent cette période d'attente pour mieux observer le batîment dans lequel ils allaient entrer.
    Cette vielle poissonerie faisait dos à la rivière Jeggare, qui était du côté est de Korvosa.
    Au premier coup d'oeil, tous pouvaient déduire qu'aucune activité légale ne s'était déroulé récemment dans ce bâtiment.
    Le mur ouest était dans le même état de délabrement que les bâtiments du quartier.
    Les planches de ce mur étaient pourries et trouées par endroit.
    Au sol, deux portes dans le même état que le mur étaient fermées pour empêcher les curieux d'observer l'intérieur du bâtiment.
    Sur la façade nord du bâtiment, une étroite galerie faite en planches menait jusqu'à la rivière.
    Un logo en forme de poisson était déssiné maladroitement sur le mur avec de la peinture rouge, couleur utilisée par Gaedrenn Lamm pour marquer ses biens.
    Le long de cette galerie, deux petites portes de service étaient elles aussi fermées et dans le même état que le bâtiment.
    Sur la façade sud du bâtiment, une galerie surélévée de 15 pieds accessible par l'intérieur de la poissonnerie menait jusqu'à la rivière.
    Sur la rivière, la galerie était connectée à un vieux bateau de pêche qui n'avait visiblement pas été utilisé récemment.\\
    \\
    Le soir venu, le groupe s'approche de la poissonnerie.\\
    \\
    \speech{Elyn}{Alors, comment allons nous entrer?}
    \\
    \speech{Pojin}{Regardez et apprenez.}
    \\
    Pojin leur fit ensuite signe de rester en retrait et se dirigea vers la plus petite des deux portes sur la façade ouest.
    Il posa l'oreille sur la porte, mais n'entendit rien.
    Il essaya ensuite d'ouvrir discrètement la porte, mais cette porte était fermée à clé.
    Il cogna donc à la porte et commença à parler de vive voix.\\
    \\
    \speech{Pojin}{
        Bonjour!
        Je désire m'entretenir avec Gaedrenn Lamm.
        Auriez-vous l'amabilité de venir m'ouvrir?
    }
    \\
    En réponse à son appel, il entendit un chien aboyer à l'intérieur.
    Il entendit ensuite un homme faire taire le chien, puis il n'entendit plus rien.
    Il cogna plusieurs fois, mais il n'entendit plus un son.
    Avant que le groupe puisse réagir, il répéta la manoeuvre sur la seconde porte du mur ouest.
    Cette fois, il n'entendit rien.
    Déçu, il fit signe au groupe de le rejoindre sur la galerie du mur nord.\\
    \\
    \speech{Elyn}{Idiot.}
    \\
    À la suite de cette tentative de Pojin, le groupe de rendit à la première porte du mur nord.
    Puisqu'il avait une longueur d'avance, Pojin en profita pour répéter son manège sur les deux autres portes qui se trouvaient devant lui.
    À la première porte, il n'entendit rien.
    À la seconde porte, plus près de la rivière, il entendit des sons de conversation, mais il ne réussit pas à comprendre ce qui était dit.
    Lorsqu'il cogna, les sons se turent.
    Il revint donc à la première porte où se trouvait le groupe qui l'avait enfin rattrapé.
    Pojin expliqua au groupe que toutes les portes étaient fermées à clé et ce qu'il avait entendu.
    Voyant que leur plan initial d'entrer discrètement était maintenant un échec monumental, Rothaul sorti sa lance et se dirigea vers la porte.
    Utilisant son arme comme levier, Rothaul força la porte affaiblie qui vola en éclats.
    À l'intérieur, le groupe entendit des petits cris de surprise.\\
    \\
    \speech{???}{Taisez-vous!}
    \\
    Ensuite, ils n'entendirent plus rien.
\end{document}
