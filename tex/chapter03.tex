\documentclass[main.tex]{subfiles}

\begin{document}
    Après ce dernier encouragement, elle se leva, pris ses cartes de tarot et quitta par la porte cachée par une tapisserie derrière elle.
    Le groupe se consulta du regard.
    Ils avaient tous une raison d'haïr Gaedrenn Lamm, mais ils ne désiraient pas la communiquer avec des étranger.
    Olrynth, qui connaissait bien la ville, se leva et fit signe au groupe de le suivre.
    Il quitta ensuite le bâtiment et se dirigea vers le 17e quai.
    Les trois autres personnages se levèrent et lui emboîtèrent le pas.
    Ils étaient tous déterminés à en finir une fois pour toute.\\
    \\
    Lorsque le groupe arriva à la poissonnerie, ils décidèrent d'attendre le soir avant d'agir.
    Ils utilisèrent cette période d'attente pour mieux observer le batîment dans lequel ils allaient entrer.
    Cette vielle poissonerie faisait dos à la rivière Jeggare, qui était du côté est de Korvosa.
    Au premier coup d'oeil, tous pouvaient déduire qu'aucune activité légale ne s'était déroulé récemment dans ce bâtiment.
    Le mur ouest était dans le même état de délabrement que les bâtiments du quartier.
    Les planches de ce mur étaient pourries et trouées par endroit.
    Au sol, deux portes dans le même état que le mur étaient fermées pour empêcher les curieux d'observer l'intérieur du bâtiment.
    Sur la façade nord du bâtiment, une étroite galerie faite en planches menait jusqu'à la rivière.
    Un logo en forme de poisson était déssiné maladroitement sur le mur avec de la peinture rouge, couleur utilisée par Gaedrenn Lamm pour marquer ses biens.
    Le long de cette galerie, deux petites portes de service étaient elles aussi fermées et dans le même état que le bâtiment.
    Sur la façade sud du bâtiment, une galerie surélévée de 15 pieds accessible par l'intérieur de la poissonnerie menait jusqu'à la rivière.
    Sur la rivière, la galerie était connectée à un vieux bateau de pêche qui n'avait visiblement pas été utilisé récemment.\\
    \\
    Le soir venu, le groupe s'approche de la poissonnerie.\\
    \\
    \speech{Elyn}{Alors, comment allons nous entrer?}
    \\
    \speech{Pojin}{Regardez et apprenez.}
    \\
    Pojin leur fit ensuite signe de rester en retrait et se dirigea vers la plus petite des deux portes sur la façade ouest.
    Il posa l'oreille sur la porte, mais n'entendit rien.
    Il essaya ensuite d'ouvrir discrètement la porte, mais cette porte était fermée à clé.
    Il cogna donc à la porte et commença à parler de vive voix.\\
    \\
    \speech{Pojin}{
        Bonjour!
        Je désire m'entretenir avec Gaedrenn Lamm.
        Auriez-vous l'amabilité de venir m'ouvrir?
    }
    \\
    En réponse à son appel, il entendit un chien aboyer à l'intérieur.
    Il entendit ensuite un homme faire taire le chien, puis il n'entendit plus rien.
    Il cogna plusieurs fois, mais il n'entendit plus un son.
    Avant que le groupe puisse réagir, il répéta la manoeuvre sur la seconde porte du mur ouest.
    Cette fois, il n'entendit rien.
    Déçu, il fit signe au groupe de le rejoindre sur la galerie du mur nord.\\
    \\
    \speech{Elyn}{Idiot.}
    \\
    À la suite de cette tentative de Pojin, le groupe de rendit à la première porte du mur nord.
    Puisqu'il avait une longueur d'avance, Pojin en profita pour répéter son manège sur les deux autres portes qui se trouvaient devant lui.
    À la première porte, il n'entendit rien.
    À la seconde porte, plus près de la rivière, il entendit des sons de conversation, mais il ne réussit pas à comprendre ce qui était dit.
    Lorsqu'il cogna, les sons se turent.
    Il revint donc à la première porte où se trouvait le groupe qui l'avait enfin rattrapé.
    Pojin expliqua au groupe que toutes les portes étaient fermées à clé et ce qu'il avait entendu.
    Voyant que leur plan initial d'entrer discrètement était maintenant un échec monumental, Rothaul sorti sa lance et se dirigea vers la porte.
    Utilisant son arme comme levier, Rothaul força la porte affaiblie qui vola en éclats.
    À l'intérieur, le groupe entendit des petits cris de surprise.\\
    \\
    \speech{???}{Taisez-vous!}
    \\
    Ensuite, ils n'entendirent plus rien.
    Elyn prit les devants et le groupe s'engagea dans l'embrasure de la porte détruite.
    À l'intérieur, le groupe trouva trois orphelins apeurés par cette entrée fracassante.
    Adossés au mur opposé, du côté sud, les enfants pointaient des fourches vers le groupe.
    Leur regard alternait entre le groupe et un quatrième enfant appuyé contre un bassin de poisson pourri, dans le coin nord-est de la pièce.
    Contrairement aux autres enfants, celui-ci, sur la gauche du groupe, n'était pas armé d'une fourche.
    Ses habits, quoique sales, étaient en bien meilleur état que les lembeaux portés par les autres enfants.
    Son visage, caché partiellement par un capuchon, révélait un sourire malicieux.
    Assumant que cet enfant était un des sbires de Gaedrenn Lamm, il s'en approcha et ne souffla que trois mots.\\
    \\
    \speech{Olrynth}{Où est Gaedrenn?}
    \\
    En guise de réponse, l'enfant rit à plein poumon, sorti une dague de ses habits et tenta de poignarder Olrynth.
    Habitué à ce genre de situation, Olrynth dévia le couteau d'une main.
    Ce mouvement brusque vit voler le capuchon de l'enfant.
    Sous le capuchon, Olrynth pu voir les traits distinctifs d'un Halfelin.
    Cet Halfelin était bien connu dans les rues de Midland.
    Il s'agissait d'Hookshank, un des «little lambs» de Gaedrenn Lamm.
    D'un tempéremment violent, cet Halfelin était sous l'influence d'une drogue différente du Frisson.
    Après cette altercation, le groupe dégaina et se prépara au combat tandis qu'Hookshank cria des ordres aux orphelins.\\
    \\
    \speech{Hookshank}{Battez-vous, sinon je vous donne en pâture à Boo.}
    \\
    Intimidés par cette menace, les orphelins encerclèrent Olrynth pour tenter le l'empaler avec leur fourche.
    Olrynth, sans réfléchir aux conséquences, s'était approché du mur nord par lequel le groupe était entré pour esquiver la dague.
    Il esquiva habilement la première et la seconde fourche, mais il glissa sur le plancher mouillé par un mélange d'eau salée, d'algues et de sang de poisson.
    Cette maladresse le forca à se retourner pour bloquer la fourche de l'orphelin derrière lui, ce qui lui valu quelques égratignures.
    Hookshank ne manqua évidemment pas cette opportunité.
    Dès qu'Olrynth leva les bras pour bloquer la fourche, il enfonca sa dague dans l'épaule d'Olrynth.
    Ayant terminé leurs préparatifs, Elyn, Podjin et Rothaul se lancèrent à sa rescousse.
    Dégainant son sabre, Elyn s'approcha des orphelins, mais ne les attaqua pas.
    Voyant que leur état mental était fragile, elle tenta de les convaincre de lâcher les armes.
    \\
    \speech{Elyn}{Cessez de vous battre! Nous ne somme ici que pour Gaedrenn et ses sbires.}
    \\
    Les attaques des orphelins cessèrent temporairement.
    Ils posèrent leur regard sur Elyn, puis sur Hookshank.
    Après une courte hésitation, ils recommencèrent à attaquer Olrynth.
    Les menaces d'Hookshank s'étaient avérées très efficaces.
    Podjin, voyant que les enfants avaient réagit à la tentative d'Elyn pour les convaincre, se mit à chanter tout en se plaçant au centre du mur ouest, devant la porte qui menait à l'extérieur.
    Cette chanson, aux paroles improvisées pour répéter les propos d'Elyn, aiguisa les réflexes de ses alliés.
    En plus de chanter, Podjin visa Hookshank avec son arc et décocha une flèche dans sa direction.
    Cette flêche manqua sa cible et se logea dans une porte derrière Hookshank qui donnait sur une autre pièce de la poissonnerie.
    Les orphelins eurent la même réaction qu'avec Elyn, mais prirent peur lorsque la flêche passa près d'eux.
    Ils lachèrent leurs fourches et se massèrent dans le coin sud-ouest de la pièce.
    Voyant cette opportunité, Rothaul enfourcha Luc, sorti sa lance et fondit sur Hookshank.
    La lance s'enfonca profondément dans le bras de ce dernier, qui cessa de rire.
    Voyant qu'il ne réussirait pas à se déberasser des intrus par lui-même, il se retourna et fonça dans la pièce adjacente par la porte derrière lui.\\
    \\
    Podjin pourchassa Hookshank, mais lorsqu'il tenta d'ouvrir la porte, celle-ci cogna dans quelque chose et se referma.
    Il entendit un grognement sourd de l'autre côté de la porte fermée.
    Pendant ce temps, Elyn fit appel à son pouvoir divin pour soigner les blessures d'Olrynth et alla voir les orphelins.
    Sortant une carte, elle leur montra le chemin jusqu'à l'église de la ville, endroit qui leur fournirait logis et sécurité.
    Voyant la possibilité de fuir la vie misérable qu'ils vivaient jusqu'à présent, les orphelins ouvrirent la porte ouest de la poissonnerie et quittèrent.
    Le groupe se conciala avant de continuer à explorer la poissonerie.
    Les deux murs intérieurs de la poissonnerie avaient également chacun une porte.
    Ils optèrent contre pourchasser Hookshank, du côté est, puisque la possibilité d'une ambuscade était trop dangereuse.
    Il ne restait donc que la porte du côté sud.
    Elyn se dirigea vers celle-ci et l'ouvrit.
    Cette pièce, contrairement à la précendente, était relativement propre.
    À ses pieds, Elyn entendit un grognement de ce qui ne pouvait être que Boo : un chien de garde à l'air vicieux et au pelage noir taché par le sang de poisson.
    Devant elle, le maître de Boo, un vieil homme maigre et visiblement faible ordonna à son animal de passer à l'attaque.
    Au même moment, Hookshank entra par une autre porte, du côté est, et s'addressa au vieillard.
    \\
    \speech{Hookshank}{Yorgen, intrus!}
    \\
    \speech{Yorgen}{Je sais, Hookshank. Ils ont même eu la brillante idée de cogner à ma porte pour m'annoncer leur arrivée.}
\end{document}
